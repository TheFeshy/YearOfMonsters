%Elvis has Entered the Building

% An open-source mystery book for Monster of the Week
% CC BY-NC-SA by DaveW

\documentclass{motw}

%============================================================================
\begin{comment}

This document generates a PDF of my mystery, Evlis has entered the building.  See the nearby readme.markdown for instructions.

An Inexperienced necromancer tries to strike it rich, but ends up ressurrecting the Confederacy.

\end{comment}
%============================================================================

\begin{document}

\appendtographicspath{{content/mysteries/Elvis has Entered the Building/images}}

% Mystery title
\Mystery{Elvis~has~Entered~the Building}{Dave W}

\imageTop[width=\paperwidth]{header2}
\imageCredit[modified]{Elvis}{}{Public Domain}

\Mhead{Concept}

More than a century ago Karl Rathburn was the most feared necromancer of the Confederacy, until his death at the hands of Union-aligned Hunters.  His distant ancestor "Chuck" Rathburn, by contrast, is a lazy high school dropout. But Chuck has "the blood," and after discovering  Karl's \emph{Necronomicon} in some old family junk, his girlfriend Daisy comes up with a way to cash in on his necromantic gifts: summon dead celebrities to create new original work.

Unfortunately Daisy's greed has made them reckless, and not every soul he summons from beyond the veil is as much fun as Elvis.  Soon he finds himself being forced to raise an entire army of undead Confederate soldiers, primed to reinstate their lost South.

\Mhead{Hook}

The daytime talk shows are all abuzz with an unusual bit of forgery gossip - a redneck named Chuck from an old family tried to pass off a "lost recording" of a never before heard Elvis song.  He was caught easily because it was fresh off of a CD, but all the Elvis experts agree: the voice, lyrics, and every other quality they cared to examine was indistinguishable from the real deal.  Could it be related to the recent Elvis sightings in the same town?

\Countdown%
{Elvis is spotted in Guntersville, Alabama around the same time Chuck tries to sell a "forgery" of a brand new Elvis song on CD}%
{A string of joyrides, breaking-and-entering, and general mischief by the ghosts of dead celebrities (who are mistaken for teenagers in costume)}%
{Chuck begins to focus on Civil War summons, and some Confederate soldiers attack the townsfolk}%
{Chuck decides to gain control of the soldiers by summoning Barton, their commander.  Instead Barton immediately takes command of them and turns on Chuck, driving him to summon more soldiers}%
{Chuck is forced to cast a complicated ritual to bring back his ancestor, Karl, who can give the confederate ghosts much greater power}%
{A reanimated Karl takes control of the confederates and storms the town, bent on re-forming the Confederacy.}%

\Mhead{Background}

Chuck Rathburn had been content to waste his life away, until he inherited the family "mansion" - a civil-war era house fallen into almost complete decay.  But it did contain one treasure:  His ancestor's \emph{Necronomicon.}  With this, he decided to grant his gold-digging girlfriend Daisy's most fervent wish: to meet Elvis Presley.

Once she saw Elvis, Daisy struck upon a way to cash in:  They'd compel the spirit to record an original Elvis song.  This backfired because while CDs might seem like "really old" recording technology to Chuck and Daisy, they were obviously not Elvis originals.

At Daisy's behest, Chuck begins summoning one celebrity after another in an attempt to create some marketable merchandise.  He and Daisy also begin stealing vintage items to add realism to these "forgeries."

Eventually, Chuck realizes the most popular antiques in his area are Civil War memorabilia, and begins summoning Confederate soldiers.  Unfortunately, his deep family ties to The South make these summonings more powerful and less controlled - and things quickly begin to spiral out of control.

\mhead{Play Notes}

You should consider asking the players for their hunter's favorite historical figure or celebrity before beginning this mystery; they may have a chance to meet them.

This mystery escalates quickly, but it is possible that the Hunters will stop Chuck before he begins the most dangerous parts (Dusk and later.)  If this happens and you have session time left, consider having your Hunters interact with ghosts of people they have lost. They'll only have until sunset/sundown, and the ghosts don't remember anything except vague impressions from their time dead.


\imageBottom[width=5in, yoffset=0.5in]{mansion4}
\imageCredit[modified]{Dilapidated Mansion}{Suzanne Crenshaw}{Public Domain}

\Mhead{Threats}

\Cast{Charles "Chuck" Rathburn (necromancer)}{minion}{cultist}

Chuck is lazy and unmotivated - if it weren't for the potent blood of Karl Rathburn running through his veins, he'd have difficulty ordering a pizza, let alone summoning the spirits of the departed.  As it is, he is able to do little more than bring them across the veil for a few hours, during which he has scant control over them at all.

This suits him just fine; he's content drinking cheap liquor and hanging out in his crumbling mansion while summoning his "party tricks" for Daisy.  But \emph{she} has plans for Chuck's powers.

\Cast{Daisy McFust (girlfriend)}{minion}{guardian}

Daisy never thought twice about Chuck, until he inherited the Mansion on the Hill.  Unfortunately, by the time she saw what terrible shape it was in, she'd already insinuated herself as his girlfriend.  Before she could find a way to bail on him, he discovered the \emph{Necronomicon}, and summoned Elvis.  Immediately seeing the monetary potential of bringing back the dead, Daisy has been pushing Chuck into ever more elaborate money-making schemes.  She will guard access and information about Chuck, the \emph{Necronomicon}, and the spirits tightly - that's her meal ticket!

\Cast{A Parade of Celebrity Ghosts Including Elvis}{monster}{trickster}

The dead celebrities Chuck summon know they will only get a few hours on Earth, and few want to spend it making nick-knacks for a high-school dropout.  Most escape the mansion, and head into what passes for "town" in Guntersville for some fun - treating it like a day-pass from prison.

Chuck is not a skilled necromancer, however, and all these spirits are quite a bit more fragile than spirits usually are.

Try to pick dead celebrities that the Hunters will recognize and find interesting.  As Daisy's favorite, Elvis may be summoned on multiple days.

\begin{mStatBlock}{Powers}
    \mStat{One Chance}{Unlike many spirits, all the ghosts Chuck summon only get a single (though mundane and realistic) manifestation.  If they are discorporated, or choose to do so, they leave the world of the living and cannot return unless re-summoned.}
\end{mStatBlock}

\begin{mStatList}{Attacks}%
    \mStat{Grave touch}{1-harm armor piercing}%
    \mStat{Other}{anything that would be appropriate for this celebrity}%
\end{mStatList}%
%
\begin{mStatList}{Weakness}%
    \mStat{Salt, holy water, prayers, etc.}{Chuck's lack of skill leaves the ghosts vulnerable to almost any stereotypical weaknesses.}%
    \mStat{Time}{The spirits will return to the afterlife at sunset or dawn each day, and are aware of this}%
\end{mStatList}%

\harmtrack{3}

\Cast{Confederate Soldiers}{monster}{beast}

\imageCredit[modified]{Confederate Soldiers}{D.C. Bettison}{Public Domain}
\imageRight[width=1.8in, margin=0.2in, position=center]{soldier1}{%
%
To Chuck's surprise, his necromancy is much more attuned to Confederate soldiers, resulting in more powerful spirits.  Unfortunately for Chuck, these soldiers have no desire to make junk for him and Daisy.  As a descendant of the original necromancer that summoned them back during the war, what they want is to make Chuck's life every bit as unpleasant as their} afterlife has been.  They begin harassing him and the townsfolk immediately, and occasionally threaten him into summon more of their colleagues.

\begin{mStatBlock}{Powers}
    \mStat{One Chance}{See \emph{A parade of Celebrity Ghosts}}
\end{mStatBlock}

\begin{mStatList}{Attacks}
    \mStat{Battlefield Terrors}{2-harm ignore armor.  When touched, roll +Cool:
        \begin{itemize}
            \item On a 10+, you witness how this ghost died, but from a detached view
            \item On a 7-9, you experience this soldier's death first hand, and it learns one of your fears in exchange
            \item On a miss, as above except you are so shaken by the experience you are unable to act for a time.  Have the hunter describe the death they just experienced in detail.
        \end{itemize}
    }
    \mStat{Confederate Musket}{2-harm far loud ammo.  A group of soldiers can fire in a volley, adding the area tag.}
\end{mStatList}

\begin{mStatList}{Weakness}
    \mStat{Civil War casualties}{All of these soldiers have been killed more than once fighting in the civil war - the original time, and again as risen corpses.  Deep in their psyche is the idea that Civil War era weapons are lethal to them.  Any such weapon, or convincing replica, should do - muskets, bayonets, sabers, cannon, etc.  A convincing cavalry charge would knock several of them back to their side of the veil.  Other attacks against them are ignored.}
    \mStat{Loyal Soldier, Lost Cause}{More than once, these soldiers have followed their commander to their deaths.  Once Barton takes charge, if he is defeated the soldiers too will return to death.}
\end{mStatList}{Weakness}

\harmtrack{5}

\Cast{Brigadier General Joseph Barton}{monster}{queen}

\imageCredit[modified]{Braxton Bragg}{U.S. Library of Congress}{Public Domain}
\imageLeft[width=1.8in, margin=0.2in, position=center]{barton2}{%
%
Brigadier General Barton was the commanding officer of the troops Chuck has been summoning.  Chuck had hoped that with Barton's help he could rein them in, but Barton is the worst of the lot.  A century of death hasn't lessened his desire to see the south triumphant, and he has one clear path to that goal:  Force Chuck to raise Karl Rathburn. With the powerful old necromancer, the South will rise (from the dead) again.

\hspace{1.5em}Barton is cruel, driven, and controlling, and immediately begins organizing his undead soldiers to achieve his goals and turning the Mansion on the Hill into a Confederate army camp} swarming with soldiers.

\begin{mStatBlock}{Powers}
    \mStat{One Chance}{See \emph{A parade of Celebrity Ghosts}}
    \mStat{Spur the men}{Barton carries a whip that his soldiers are all too familiar with.  He can cause any of his men he can see 1 harm to spur them into a frenzied attack increasing the harm of their attacks by one ongoing.}
\end{mStatBlock}

\begin{mStatBlock}{Attacks and Weaknesses}
    \mStat{Civil War Gost}{Barton has the same attacks and weaknesses as the other soldiers; he's just made of sterner stuff}
\end{mStatBlock}

\begin{mStatBlock}{Armor}
    \mStat{Ghostly Resilience}{1-armor}
\end{mStatBlock}

\harmtrack{12}

\Cast{Karl Rathburn (weakened spirit)}{minion}{cultist}

Karl Rathburn's power was so great that, according to the few who know such secret histories, he would have personally put off the defeat of the South by another year.

But he \emph{was} stopped, and in death has paid a terrible price for his necromancy.  He is made to suffer for every bit of power he used - and also, for every bit his descendant Chuck uses.  So when Chuck first discovers his powers, Karl tries to put a stop to it.   He manages to siphon off enough necromantic energy from Chuck to manifest as a weakened spirit (see celebrity spirits, but without the \emph{One Chance} power, and only able to manifest as a hazy outline.) He will try to find ways to thwart Chuck so that he can rest in relative peace.  Unfortunately, his best bet for achieving this is the Hunters, but he has no love at all for hunters.

His plans change drastically if he is reanimated fully at \emph{Nightfall}.  As do his form and powers; so much so that he gets a second \emph{Threats} entry:

\Cast{Karl Rathburn (The Unliving Necromancer)}{monster}{sorcerer}

Use these stats for Karl after the ritual at \emph{Nightfall}

Unlike the other spirits, Karl has been raised bodily from his family crypt.  He appears as a skeleton in dramatic, flowing robes with ghostly, translucent features superimposed over the bones.

Once reanimated, Karl will abandon his plans to stop Chuck.  Instead, his focus will be resuming the power struggle he once had with Barton.  The two of them were rivals, forced to work together by greater circumstances, and that is exactly the power dynamic they resume in undeath.  Their collective goal is nothing less than raising the Confederacy from the dead as a necromantic kingdom - but the rule of that kingdom is very much in dispute.

\imageCredit{dark side guy}{publicdomainvectors}{Public Domain}
\imageBottom[width=3in, yoffset=0.4in]{karl1}

\begin{mStatList}{Powers}
    \mStat{Unliving Necromancer}{Karl regenerates any harm done to him.}
\end{mStatList}

\begin{mStatList}{Moves}
    \mStat{Voidcall}{Karl can bring one or two of Chuck's Confederate soldiers to himself with a wave of his hand, or summon new ones with slightly more effort}
\end{mStatList}

\begin{mStatBlock}{Attacks}
    \mStat{Bale-fire}{2-harm near ignore armor magic area}
\end{mStatBlock}

\begin{mStatBlock}{Weakness}
    \mStat{Necronomicon}{Karl has not had the time to create a proper phylactery, so for now his family's ancient \emph{Necronomicon} is his reservoir.  Once it is destroyed, Karl will be vulnerable to any form of harm and will not regenerate.}
\end{mStatBlock}

\Mhead{Bystanders and Locations}

\Cast{Dave Harding (Antique shop owner)}{bystander}{witness}

\imageCredit{Elvis Impersonator}{}{CC0}
\imageRight[width=1in, margin=0.5in, position=center]{dave2}{
If the hunters ask around town about Elvis, sooner or later they will be directed to Dave.  Dave has a secret even he didn't know: he's the illegitimate son of Elvis Presley, and has enough of his look that the townspeople will automatically think of him when the topic comes up.  Dave has had a very strange week - first, he met his (dead) father for the first time.  Then his antique and memorabilia shop was robbed two nights in a row.}

\Cast{Sasha Beaumont (Elvis Expert)}{bystander}{gossip}

She is in town on behalf of an undisclosed wealthy collector to examine the Elvis recording.  She's been giving interviews to papers and entertainment reporters about the amazing accuracy of the recording: \emph{"Elvis didn't record this song, but if he did it would sound exactly like this."}

She is also drawn to Dave Harding, due to his resemblance to her area of expertise.

\Cast{Professor Joanne Ivey (Museum Curator and Historian)}{bystander}{detective}

Forty years ago, Ivey's dissertation on \emph{"The use of necromancy and other magics in the defense of the South"} nearly got her expelled from academia, before she recanted it and confessed it was a "joke."  Her taking a position in the Guntersville museum is not an accident; even after all these years she still believes that there were major necromantic activities in the Confederacy that happened in or around this town.  While she hasn't shared this with anybody, she has been diligently, if quietly, researching it.

\Cast{Colonel Randy King, Retired (Civil War reenactor)}{bystander}{helper}

King was always a fan of military history, and the Civil War was always his favorite war.  When he retired, he took up Civil War reenactment as almost a full-time job.  He will be the only one in town enthused by the arrival of civil war ghosts run amok; it's his chance to combine his military career with his reenactment passion.

He can regularly be found debating the importance of occult symbolism to the Confederacy with Professor Ivey.

\Cast{Councilwoman Abigail Rodgers}{bystander}{official}

Mrs. Rodgers is a stern woman who has used her blunt personality to become de-facto chairwoman of the town council.  She likes Guntersville to be run \emph{"just so"} and is quite unhappy about all the recent disruptions - even if they \emph{have} brought more tourists to town.

\Cast{The Mansion on the Hill}{location}{lab}

This building dates back to the Civil War itself, and while it has \emph{probably} had maintenance done since then, it doesn't look it now.  Chuck inherited it from distant relatives, but to Chuck's small ambitions, even a crumbling mansion seems fantastic.

The grounds contain many crumbling rooms, a family crypt, and the recently unearthed lab of Karl Rathburn.  The upper mansion is dangerous to those unfamiliar with it, with rotting floors and ceilings that occasionally collapse.

\begin{mStatBlock}{Custom Move}
    \mStat{Crumble}{Whenever it would be interesting, part of the mansion crumbles.  This could add humor, or reveal information, or expose the hunters to new monsters, or even harm them directly}
\end{mStatBlock}

\imageBottom[width=\paperwidth]{battle3}
\imageCredit[modified]{Hancock at Gettysbug}{Thure de Thulstrup}{Public Domain}

\Cast{Guntersville Confederate Museum}{location}{crossroads}

It is here that Professor Ivey researches Confederate occultism, and it has a huge store of Confederate artifacts.  It's the main tourist attraction in town, and is the default meeting place of many of the town's civil war reenactors.  It's a bit of a mystery as to why there is such a big crowd of civil war reenactors living in Guntersville, as the town hosted no major battles.

\Cast{Old Dave's Antiques and Memorabilia}{location}{hub}

This store is directly across from the Confederate Museum, and specializes in selling Civil War memorabilia to its visitors.  Dave Harding is its owner, and it has recently been robbed (twice in a row) by Chuck and Daisy, who are searching for more authentic tools with which their ghosts can make new historical artifacts to be sold.

\Cast{The Confederate Camp (Dusk and later)}{location}{maze}

\imageRight[width=2in, margin=.5in, side=right, position=center]{camp3}{%
%
Once Barton arrives on scene, he whips the soldiers into a regiment again (literally, if necessary) including putting up a camp around the Mansion on the Hill.  He's not clear on why ghosts would need tents or cook fires or all the rest, but his soldiers learned over a century ago not to question his orders.

It's crawling with confederate soldier ghosts, but they are more focused on not angering their commander than defending the camp against Union soldiers who have been gone a century anyway.}

\imageCredit[modified]{Confederate Camp}{NARA}{Public Domain}

\makeImageCredits[cc-by-nc-sa]

\end{document}